\subsection{Version your code}

The first step is to learn how to version your own code.
In this tutorial, we will run Git from the command line of the Unix shell.
Thus we expect readers are already comfortable with navigating a filesystem and running basic commands in such an environment.
You can find directions for installing Git for the operating system running on your computer by following one of the links provided in Table 1.
There are many graphical user interfaces (GUIs) available for running Git (Table 1), which we encourage you to explore, but learning to use Git on the command line is necessary for performing more advanced operations and using Git on a remote machine.

For the purposes of this tutorial, imagine you have a folder in your home directory named \verb|thesis|, which contains three files.
\verb|process.sh| runs some common bioinformatics tools on your raw data, \verb|clean.py| removes bad samples and combines the data into a matrix, and \verb|analyze.R| runs a statistical test and plots the result.

If you have just installed Git, the first thing you need to do is provide some information about yourself, since it records who makes each change to the file(s).
Set your name and email by running the following lines, but replacing "First Last" and "user@domain" with your full name and email address, respectively.

\begin{lstlisting}
$ git config --global user.name "First Last"
$ git config --global user.email "user@domain"
\end{lstlisting}

To start versioning your code with Git, navigate to your newly created or existing project directory (in this case, \verb|~/thesis|).
Start tracking your code by running the command \verb|git init|, which initializes a new Git repository in the current folder.
A repository refers to the current version of the tracked files as well as all the previously saved versions (Box 1).

\begin{lstlisting}
$ cd ~/thesis
$ ls
analyze.R clean.py process.sh
$ git init
Initialized empty Git repository in ~/thesis/.git/
\end{lstlisting}

Now you are ready to start tracking your code (\ref{fig:Fig1}).
Conceptually, Git saves snapshots of the changes you make to your files whenever you instruct it to.
For instance, after you edit a script in your text editor, you save the updated script to your thesis folder.
If you tell Git to save a shapshot of the updated document, then you will have a permanent record of the file in that exact version even if you make subsequent edits to the file.
In the Git framework, any changes you have made to a script, but have not yet recorded as a snapshot with Git, reside in the working diretory (\label{fig:Fig1}).
To follow what Git is doing as you record the initial version of your files, use the informative command \verb|git status|.

\begin{lstlisting}
$ git status
On branch master

Initial commit

Untracked files:
  (use "git add <file>..." to include in what will be committed)

        analyze.R
        clean.py
        process.sh

nothing added to commit but untracked files present (use "git add" to track)
\end{lstlisting}

There are a few key things to notice from this output.
First, the three scripts are recognized as untracked files because you have not told git to take snapshots of anything yet.
Second, the word "commit" means "a version of the code", e.g. "the figure was generated using the commit from yesterday" (Box 1).
This word can also be used as a verb, in which case it means "to save", e.g. "to commit a change."
Lastly, it explains how you can start tracking your files.
You need to use the command \verb|git add|.
Add the file \verb|process.sh|.

\begin{lstlisting}
$ git add process.sh
\end{lstlisting}

And check its new status.

\begin{lstlisting}
$ git status
On branch master

Initial commit

Changes to be committed:
  (use "git rm --cached <file>..." to unstage)

        new file:   process.sh

Untracked files:
  (use "git add <file>..." to include in what will be committed)

        analyze.R
        clean.py
\end{lstlisting}

Now the file \verb|process.sh| has been added to the staging area, while both \verb|clean.py| and \verb|analyze.R| remain unstaged.
Adding a file to the staging area will result in the changes to that file being included in the next commit, or snapshot of the code (\label{fig:Fig1}).
As an analogy, adding files to the staging area is like putting things in a box to mail off, and committing is like putting the box in the mail.

Since this will be the first commit, or first version of the code, add the other two files to the staging area as well.
Then create the first commit using the command \verb|git commit|.

\begin{lstlisting}
$ git add clean.py analyze.R
$ git commit -m "Add initial version of thesis code."
[master (root-commit) 660213b] Add initial version of thesis code.
 3 files changed, 25 insertions(+)
 create mode 100644 analyze.R
 create mode 100644 clean.py
 create mode 100644 process.sh
\end{lstlisting}

Notice the flag \verb|-m| was used to pass a message for the commit.
This message describes the changes that have been made to the code and is required.
If you do not pass a message at the command line, the default text editor for your system will open so you can enter the message.
You have just performed the typical development cycle with Git:
make some changes, add updated files to the staging area, and commit the changes as a snapshot once you are satisfied with them.

Since Git records all of the commits, you can always look through the complete history of a project.
To view the record of your commits, use the command \verb|git log|.
For each commit, it lists the the unique identifier for that revision, author, date, and commit message.

\begin{lstlisting}
$ git log
commit 660213b91af167d992885e45ab19f585f02d4661
Author: First Last <user@domain>
Date:   Sun Mar 29 14:52:05 2015 -0500

    Add initial version of thesis code.
\end{lstlisting}

The commit identifier can be used to compare two different versions of a file, restore a file to a previous version from a past commit, and even retrieve tracked files if you accidentally delete them.

Now you are free to make changes to the files knowing that you can always revert them to the state of this commit by referencing its identifier.
As an example, edit \verb|clean.py|.
Here is the current top of the file.

\begin{lstlisting}
$ head clean.py
#!/usr/bin/env python

# Thesis project:
# Remove bad samples.
# Export clean data as a matrix.

# Usage: clean.py input [input ...] > data_clean.py

import sys
import os
\end{lstlisting}

Update the description to explicitly define which samples are removed.

\begin{lstlisting}
$ head clean.py
#!/usr/bin/env python

# Thesis project:
# Remove samples with more than 5% missing data.
# Export clean data as a matrix.

# Usage: clean.py input [input ...] > data_clean.py

import sys
import os
\end{lstlisting}

You can view all the differences between the current version and last committed version of the file by running the command \verb|git diff|.

\begin{lstlisting}
$ git diff
diff --git a/clean.py b/clean.py
index c1fcad7..c0bfe5c 100644
--- a/clean.py
+++ b/clean.py
@@ -1,7 +1,7 @@
 #!/usr/bin/env python
 
 # Thesis project:
-# Remove bad samples.
+# Remove samples with more than 5% missing data.
 # Export clean data as a matrix.
 
 # Usage: clean.py input [input ...] > data_clean.py
\end{lstlisting}

The new line starts with \verb|+| and the previous line starts with \verb|-|.
You can ignore the first five lines of output because they are directions for other software programs that can merge changes to files.
If you wanted to save this edit, you could add \verb|clean.py| to the staging area using \verb|git add| and then commit the change using \verb|git commit|, as you did above.
Instead, this time restore the last committed version of the file using the command \verb|git checkout|.

\begin{lstlisting}
$ git checkout -- clean.py
$ git diff
\end{lstlisting}

Now \verb|git diff| returns no output because \verb|git checkout| reverted \verb|clean.py| to the version in the last commit.
And this ability to revert to past versions of a file is not limited to just the last commit.
If you had committed multiple changes to the file \verb|clean.py| and then decided you wanted the original version, you could replace the argument \verb|--| with the commit identifier (we only need to specify the first few characters for it to be unique).
The \verb|--| used above was simply a placeholder for the first argument because by default \verb|git checkout| restores the most recent version of the file.

\begin{lstlisting}
$ git checkout 660213b clean.py
\end{lstlisting}

There are also more advanced options for reverting history that we will not cover in this quick introduction.

At this point, you have learned the commands needed to version your code with Git.
Thus you already have the benefits of being able to make edits to files without copying them first, to create a record of your changes with accompanying messages, and to revert to previous versions of the files if needed.
Now you will always be able to recreate past results that were generated with previous versions of the code and see the exact changes you have made over the course of a project.