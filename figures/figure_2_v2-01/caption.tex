\textbf{Figure 2. Working with both a local and remote repository as a single user.} A) On your local machine you have a repository that you've committed to (commit ID d75es). B) On github, you created a new repository called "thesis". This repository is currently empty and not linked to the repo on your local machine. C) The command \verb|git remote add | connects your local repository to your remote repository. The remote repository is still empty, however, because we have not pushed any content to it. D) All the local commits are sent to the remote repository by using the command \verb|git push|. Only files that have been committed will appear in the remote repository. E) Several more rounds of updating scripts and committing are done on the local computer (commit ID f658t and then commit ID xv871). These commits have not yet been pushed to the remote repository, so only the previously pushed commit is in the remote repo (commit ID d75es). F) To sync the local and remote repositories, \verb|git push| the two new commits to the remote repository. The local and remote repositories now contain the same files and commit histories. 