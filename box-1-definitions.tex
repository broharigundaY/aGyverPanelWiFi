\subsection*{Box 1: Definitions}

\begin{itemize}
\item \textbf{Version Control System (VCS)}: \textit{(noun)} a program that tracks changes to specified files over time and maintains a library of all past versions of those files
\item \textbf{Git}: \textit{(noun)} a version control system
\item \textbf{repository (repo)}: \textit{(noun)} folder containing all tracked files as well as the version control history
\item \textbf{commit}: \textit{(noun)} a snapshot of changes made to the staged file(s); \textit{(verb)} to save a snapshot of changes made to the staged file(s)
\item \textbf{stage}: \textit{(noun)} the staging area holds the files to be included in the next commit; \textit{(verb)} to mark a file to be included in the next commit
\item \textbf{track}: \textit{(noun)} a tracked file is one that is recognized by the Git repository
\item \textbf{branch}: \textit{(noun)} a parallel version of the files in a repository (Box 7)
\item \textbf{local}: \textit{(noun)} the version of your repository that is stored on your personal computer
\item \textbf{remote}: \textit{(noun)} the version of your repository that is stored on a remote server, for instance on GitHub
\item \textbf{clone}: \textit{(verb)} to create a local copy of a remote repository on your personal computer
\item \textbf{fork}: \textit{(noun)} a copy of another user's repository on GitHub; \textit{(verb)} to copy a repository, for instance from one user's GitHub account to your own
\item \textbf{merge}: \textit{(verb)} to update files by incorporating the changes introduced in new commits
\item \textbf{pull}: \textit{(verb)} to retrieve commits from a remote repository and merge them into a local repository
\item \textbf{push}: \textit{(verb)} to send commits from a local repository to a remote repository
\item \textbf{pull request}: \textit{(noun)} a message sent by one GitHub user to merge the commits in their remote repository into another user's remote repository
\end{itemize}
