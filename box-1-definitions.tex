\subsection{Box 1: Definitions}

\begin{itemize}
\item \textbf{git}: \textit{(noun)} a version control system; \textit{(verb)} command used to run all version control processes
\item \textbf{repository (repo)}: \textit{(noun)} folder containing all items to be version controlled as well as the version control history
\item \textbf{commit}: \textit{(noun)} a snapshot of changes made to a staged file or files; \textit{(verb)} to save a snapshot of changes made to staged files 
\item \textbf{branch}: \textit{(noun)} a parallel version of your repository which separately version controlled
\item \textbf{local}: \textit{(noun)} the version of your repository that is stored on your personal computer
\item \textbf{remote}: \textit{(noun)} the version of your repository that is stored on the internet, for instance on github.com
\item \textbf{clone}: \textit{(noun)} a copy of a remote repository, made on your personal computer; \textit{(verb)} to create a copy of a remote repository on your personal computer
\item \textbf{fork}: \textit{(noun)} a copy of someone else's remote repository into your account; \textit{(verb)} to copy someone else's remote repository into your account
\item \textbf{merge}: \textit{(verb)} to combine different versions of files from two branches into one branch
\item \textbf{pull}: \textit{(verb)} to incorporate the commits made to a remote repository that are not in a local repository automatically
\item \textbf{push}: \textit{(verb)} to send commits from a local repository to a remote repository
\item \textbf{pull request}: \textit{(noun)} a message sent by one user to merge their remote repository into another users remote repository
\end{itemize}