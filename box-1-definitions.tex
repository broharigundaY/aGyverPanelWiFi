\subsection{Box 1: Definitions}

\begin{itemize}
\item \textbf{Git}: \textit{(noun)} a version control system
\item \textbf{repository (repo)}: \textit{(noun)} folder containing all tracked files as well as the version control history
\item \textbf{commit}: \textit{(noun)} a snapshot of changes made to the staged file(s); \textit{(verb)} to save a snapshot of changes made to the staged file(s) 
\item \textbf{branch}: \textit{(noun)} a parallel version of the files in a repository (Box 2)
\item \textbf{local}: \textit{(noun)} the version of your repository that is stored on your personal computer
\item \textbf{remote}: \textit{(noun)} the version of your repository that is stored on the internet, for instance on GitHub
\item \textbf{clone}: \textit{(verb)} to create a local copy of a remote repository on your personal computer
\item \textbf{fork}: \textit{(noun)} a copy of a repository; \textit{(verb)} to copy a repository
\item \textbf{merge}: \textit{(verb)} to update files by incorporating the changes introduced in new commits
\item \textbf{pull}: \textit{(verb)} to retrieve commits from a remote repository and merge them into a local repository
\item \textbf{push}: \textit{(verb)} to send commits from a local repository to a remote repository
\item \textbf{pull request}: \textit{(noun)} a message sent by one user to merge the commits in their remote repository into another user's remote repository
\end{itemize}