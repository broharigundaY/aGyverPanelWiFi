\subsection{Contribute to other projects}

Lots of scientific software is hosted online in Git repositories.
Now that you know the basics of Git, you can directly contribute to the scientific software you use for your research.
From a small contribution like fixing a typo in the documentation to a larger change such as fixing a bug, it is empowering to be able to improve the software used by you and many other scientists.

When contributing to a larger project with many contributors, you will not be able to push your changes with \verb|git push| directly to the repository.
Instead you will first need to create your own copy of the repository.
Your copy of the repository is referred to as a fork.
You can fork any repository on GitHub by clicking the button "Fork" on the top right of the screen.

Once you have a fork of a repository, you can clone it and make changes just like a repository you created yourself.
Let's say we created a fork of the hypothetical repository, "cool_software", so that we could fix a typo we found in the directions in the README file.
In order to make changes, we first download it with \verb|git clone|.

\begin{lstlisting}
git clone https://github.com/scientist123/cool_project.git
\end{lstlisting}

After making the edits we want, we can add, commit, and push the changes back to our remote repository on GitHub.

\begin{lstlisting}
git add README
git commit -m "Fix typo in documentation."
git push origin master
\end{lstlisting}

Currently the typo is fixed in our fork of cool_project.
To merge this change into the main repository that is owned by the creator of the software, we send a pull request using the GitHub interface (Pull request -> New pull request -> Create pull request).
After the pull request is created, the owner of the original repository can review our change.
If she approves of the change, she can merge it into the main repository.
