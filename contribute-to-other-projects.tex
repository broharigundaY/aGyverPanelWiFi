\subsection{Contribute to other projects}

Lots of scientific software is hosted online in Git repositories.
Now that you know the basics of Git, you can directly contribute to the scientific software you use for your research.
From a small contribution like fixing a typo in the documentation to a larger change such as fixing a bug, it is empowering to be able to improve the software used by you and many other scientists.

When contributing to a larger project with many contributors, you will not be able to push your changes with \verb|git push| directly to the repository.
Instead you will first need to create your own copy of the repository.
Your copy of the repository is referred to as a fork.
You can fork any repository on GitHub by clicking the button "Fork" on the top right of the screen.
