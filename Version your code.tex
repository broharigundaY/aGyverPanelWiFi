\subsection{Version your code}

The first step is to learn how to version your own code. In this tutorial, we will run Git from the command line of the Unix shell. Thus we expect readers are already comfortable with navigating a filesystem and running basic commands in such an environment. There are many graphical user interfaces (GUIs) available for running Git (Table x), which we encourage you to explore, but learning to use Git on the command line is necessary for performing more advanced operations and using Git on a remote server. For the purposes of this tutorial, we'll imagine we have a subdirectory in our home directory named \verb|thesis|, which contains three files. \verb|process.sh| runs some common bioinformatics tools on our raw data, \verb|clean.py| removes bad samples and combines the data into a matrix, and \verb|analyze.R| runs a statistical test and plots the result.

To start versioning your code with Git, navigate to your newly created or existing project directory (in this case, \verb|~/thesis|).  Start tracking your code by running the command \verb|git init|, which \textit{init}iates a new Git repository.

\begin{lstlisting}
$ cd ~/thesis
$ ls
analyze.R clean.py process.sh
$ git init
Initialized empty Git repository in ~/thesis/.git/
\end{lstlisting}

Next you'll need to give Git some information about yourself, since it records who makes each change to the file(s). Provide your name and email by running the following lines, but replacing "First Last" and "user@domain" with your full name and email address, respectively.

\begin{lstlisting}
$ git config --global user.name "First Last"
$ git config --global user.email "user@domain"
\end{lstlisting}

Now you're ready to start tracking your code.
This requires a basic understanding of how Git tracks your files and the edits you make to them (Figure 1).
Unsaved files and/or edits are in the working directory.
This is similar to how you have worked before, i.e. edits to files in the working directory are  permanent and previous versions are not saved (unless you explicitly direct Git to do so).
To follow what Git is doing as we save the initial version of our files, we'll use the informative command \verb|git status|.

\begin{lstlisting}
$ git status
On branch master

Initial commit

Untracked files:
  (use "git add <file>..." to include in what will be committed)

        analyze.R
        clean.py
        process.sh

nothing added to commit but untracked files present (use "git add" to track)
\end{lstlisting}

There are a few key things to notice from this output.
First, notice that the three scripts are recognized as untracked files because we have not saved anything yet.
Second, notice the use of the word "commit", which is an essential Git term.
When used as a verb, it means "to save", e.g. "to commit a change."
When used as a noun, it means "a version of the code", e.g. "the figure was generated using the commit from yesterday."
Lastly, notice that it explains how we can start tracking our files.
We need to use the command \verb|git add|.
Let's add the file \verb|process.sh|.

\begin{lstlisting}
$ git add process.sh
$ git status
On branch master

Initial commit

Changes to be committed:
  (use "git rm --cached <file>..." to unstage)

        new file:   process.sh

Untracked files:
  (use "git add <file>..." to include in what will be committed)

        analyze.R
        clean.py
\end{lstlisting}

Now the file \verb|process.sh| has been moved to the staging area, while both \verb|clean.py| and \verb|analyze.R| remain in the working directory.
Adding a file to the staging area will result in the changes to that file being included in the next commit, or snapshot of the code (Figure 1).
Since this will be the first commit, or version of the code, let's add the other two files as well.
Then we'll create the first commit using the command \verb|git commit|.

\begin{lstlisting}
git add clean.py analyze.R
git commit -m "Add initial version of thesis code."
\end{lstlisting}

Notice we used the flag \verb|-m| to pass a message for the commit.
This message describes the changes that have been made to the code and is required.
If you do not pass a message at the command line, you will be entered into the default text editor for your system, usually \verb|vi| or \verb|emacs|, to enter the message.