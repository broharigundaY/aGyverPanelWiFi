\subsection{Introduction to version control}

A few key problems arise when writing large amounts of code. First, you often need to experiment with new ideas, such as adding new features or increasing the speed of a bottleneck, but you don't want to risk breaking the currently working code. The simplest solution is to copy the script before making new edits. However, this can quickly become a problem because it clutters your filesystem with uniformative filenames, e.g. \verb|analysis.sh|, \verb|analysis_02.sh|, \verb|analysis_final_03.sh|, etc. It is difficult to remember the differences between the version of the code, and more importantly which version you used to produce specific results, especially if you return to the code months later. Second, you will likely need to incorporate changes from collaborators. If you email the code to multiple collaborators, you will have to manually incorporate all the changes each of them sends. Fortunately, software engineers have already developed software to manage these issues: version control.