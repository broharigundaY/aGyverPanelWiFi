\subsubsection{Box 4: What \textit{not} to version control}

You \textit{can} version control any file that you put in a git repository, whether it is text-based, an image, or giant data files. However, just because you \textit{can} version control something, does not mean you \textit{should}. Git works great for plain text based documents such as your scripts or your manuscript if written in latex or markdown. This is because for text files, git saves the entire file only the first time you commit it and then saves just your changes with each commit. This takes up very little space and git has the capabilities to compare between versions built in (using \verb|git diff|). You can commit a non-text file, but a full copy of the file will be saved with each commit. Over time, you may find the size of your repository growing very quickly. A good rule of thumb is to version control anything text based: your scripts or manuscripts if they are written in plain text. Things \textit{not} to version control are large data files that never change, binary files (including Word and Excel documents), and the output of your code. 
