\subsection{Share your code}

Once you have a directory of version controlled files, referred to as a repository, you can share it with your collaborators and the wider scientific community by putting your code online.
This is made easier if you use one of the many online services that host Git repositories (Table x), e.g. GitHub.

To begin using GitHub, you'll first need to sign up for an account.
For the examples in this tutorial, we'll use the fake username "scientist123".
Next you'll need to choose the option to "Create a new repository".
We'll call it "thesis" because that is the directory name containing the files, but this is not a requirement.
Also, now that the code will be existing in multiple places, we need to learn some more terminology.
A local repository refers to code that is stored on the machine you are using, e.g. your laptop; whereas, a remote repository refers to the code that is hosted online.
Thus we have just created a remote repository.

